\usepackage{float}
\usepackage{geometry}
\usepackage[T1]{fontenc}

\usepackage{hyperref}
\usepackage{graphicx}
\usepackage{subcaption}
\usepackage{emptypage}
\usepackage{titling}

\usepackage{enumitem}

\usepackage{amsmath}
\usepackage{amsthm}
\usepackage{amssymb}
\usepackage{systeme}
\usepackage{spalign}
\usepackage{siunitx}
\usepackage{cancel}

\usepackage{tikz}
\usetikzlibrary{calc,patterns,angles,quotes}
\usetikzlibrary{automata}
\usetikzlibrary{positioning}
\usetikzlibrary{babel}

\usepackage{pgfplots}
\pgfplotsset{compat=1.18} 

% Theorem definitions based on Gilles Castel's work: 
% Source: https://github.com/gillescastel/university-setup
\usepackage{thmtools}
\usepackage[framemethod=TikZ]{mdframed}
\mdfsetup{skipabove=1em,skipbelow=0em, innertopmargin=5pt, innerbottommargin=6pt}

\theoremstyle{definition}

\declaretheoremstyle[headfont=\bfseries\sffamily, bodyfont=\normalfont, mdframed={ nobreak } ]{thmbox}
\declaretheoremstyle[headfont=\bfseries\sffamily, bodyfont=\normalfont]{notestyle}
\declaretheoremstyle[headfont=\bfseries\sffamily, bodyfont=\normalfont, numbered=no, mdframed={ rightline=false, topline=false, bottomline=false, }, qed=\qedsymbol ]{thmproofbox}
\declaretheoremstyle[headfont=\bfseries\sffamily, bodyfont=\normalfont, numbered=no, mdframed={ nobreak, rightline=false, topline=false, bottomline=false } ]{thmexplanationbox}

\declaretheorem[numberwithin=chapter, style=thmbox, name=Definición]{definition}
\declaretheorem[sibling=definition, style=thmbox, name=Corolario]{corollary}
\declaretheorem[sibling=definition, style=thmbox, name=Proposición]{prop}
\declaretheorem[sibling=definition, style=thmbox, name=Teorema]{theorem}
\declaretheorem[sibling=definition, style=thmbox, name=Lema]{lemma}

\declaretheorem[numbered=no, style=thmexplanationbox, name=Explicación]{explanation}
\declaretheorem[numbered=no, style=thmproofbox, name=Demostración]{replacementproof}
\declaretheorem[style=notestyle, name=Ejercicio]{ex}
\declaretheorem[style=notestyle,  numbered=no, name=Ejemplo]{eg}
\declaretheorem[style=notestyle, numbered=no, name=Nota]{note}

\renewenvironment{proof}[1][\proofname]{\begin{replacementproof}}{\end{replacementproof}}

\AtEndEnvironment{eg}{\null\hfill$\lozenge$}%
\AtEndEnvironment{ex}{\null\hfill$\blacklozenge$}%

\newtheorem*{notation}{Notación}
\newtheorem*{solution}{Solución}
\newtheorem*{observation}{Observación}
\newtheorem*{property}{Propiedad}

% For code syntax highlighting
\usepackage{listings}

% Flexoki colors
% Source: https://stephango.com/flexoki
\definecolor{bgcolor}{RGB}{255, 252, 240}
\definecolor{txcolor}{RGB}{16, 15, 15}
\definecolor{tx2color}{RGB}{111, 110, 105}
\definecolor{grcolor}{RGB}{102, 128, 11}
\definecolor{recolor}{RGB}{175, 48, 41}
\definecolor{cycolor}{RGB}{36, 131, 123}

% Allow for some special characters on listings
\lstset{literate=
    {á}{{\'a}}1 {é}{{\'e}}1 {í}{{\'i}}1 {ó}{{\'o}}1 {ú}{{\'u}}1
    {Á}{{\'A}}1 {É}{{\'E}}1 {Í}{{\'I}}1 {Ó}{{\'O}}1 {Ú}{{\'U}}1
    {à}{{\`a}}1 {è}{{\`e}}1 {ì}{{\`i}}1 {ò}{{\`o}}1 {ù}{{\`u}}1
    {À}{{\`A}}1 {È}{{\`E}}1 {Ì}{{\`I}}1 {Ò}{{\`O}}1 {Ù}{{\`U}}1
    {ä}{{\"a}}1 {ë}{{\"e}}1 {ï}{{\"i}}1 {ö}{{\"o}}1 {ü}{{\"u}}1
    {Ä}{{\"A}}1 {Ë}{{\"E}}1 {Ï}{{\"I}}1 {Ö}{{\"O}}1 {Ü}{{\"U}}1
    {â}{{\^a}}1 {ê}{{\^e}}1 {î}{{\^i}}1 {ô}{{\^o}}1 {û}{{\^u}}1
    {Â}{{\^A}}1 {Ê}{{\^E}}1 {Î}{{\^I}}1 {Ô}{{\^O}}1 {Û}{{\^U}}1
    {ã}{{\~a}}1 {ẽ}{{\~e}}1 {ĩ}{{\~i}}1 {õ}{{\~o}}1 {ũ}{{\~u}}1
    {Ã}{{\~A}}1 {Ẽ}{{\~E}}1 {Ĩ}{{\~I}}1 {Õ}{{\~O}}1 {Ũ}{{\~U}}1
    {œ}{{\oe}}1 {Œ}{{\OE}}1 {æ}{{\ae}}1 {Æ}{{\AE}}1 {ß}{{\ss}}1
    {ű}{{\H{u}}}1 {Ű}{{\H{U}}}1 {ő}{{\H{o}}}1 {Ő}{{\H{O}}}1
    {ç}{{\c c}}1 {Ç}{{\c C}}1 {ø}{{\o}}1 {Ø}{{\O}}1 {å}{{\r a}}1 {Å}{{\r A}}1
    {€}{{\euro}}1 {£}{{\pounds}}1 {«}{{\guillemotleft}}1
    {»}{{\guillemotright}}1 {ñ}{{\~n}}1 {Ñ}{{\~N}}1 {¿}{{?`}}1 {¡}{{!`}}1 
}

\lstdefinestyle{flexokistyle}{
    commentstyle=\color{tx2color},
    keywordstyle=\color{recolor},
    numberstyle=\ttfamily\color{txcolor},
    stringstyle=\color{grcolor},
    identifierstyle=\color{cycolor},
    basicstyle=\ttfamily\scriptsize,
    breakatwhitespace=false,
    breaklines=true,
    captionpos=b,
    keepspaces=true,
    numbers=left,
    numbersep=5pt,
    showspaces=false,
    showstringspaces=false,
    showtabs=false,
    tabsize=4,
    frame=single,
}
\lstset{style=flexokistyle}

\renewcommand\qedsymbol{$\blacksquare$}


% Acronyms
\usepackage[automake,acronym]{glossaries-extra}
\setabbreviationstyle[acronym]{long-short}
\newacronym{fg}{F.G.}{Francisco Galindo}
\makeglossaries

% Customize headers
\usepackage{fancyhdr}
\pagestyle{fancy}

\fancyhf{}
\newcommand{\changefont}{%
    \fontsize{9}{11}\selectfont
}
\fancyhead[LO]{\changefont \slshape \rightmark} %section
\fancyhead[RE]{\changefont \slshape \leftmark} %chapter
\fancyhead[LE, RO]{\changefont \thepage} %chapter


% 'dedication' environment: To add a dedication paragraph at the start of book
% Source: http://www.tug.org/pipermail/texhax/2010-June/015184.html
\newenvironment{dedication}
{
    \cleardoublepage
    \thispagestyle{empty}
    \vspace*{\stretch{1}}
    \hfill\begin{minipage}[t]{0.66\textwidth}
        \raggedright
    }
    {
    \end{minipage}
    \vspace*{\stretch{3}}
    \clearpage
}

% Chapter quote at the start of chapter
% Source: http://tex.stackexchange.com/a/53380
\makeatletter
\newenvironment{chapquote}[2][2em]
{\setlength{\@tempdima}{#1}%
    \def\chapquote@author{#2}%
    \parshape 1 \@tempdima \dimexpr\textwidth-2\@tempdima\relax%
\itshape}
{\par\normalfont\hfill--\ \chapquote@author\hspace*{\@tempdima}\par\bigskip}
\makeatother

% Add syllogism definition
% Source: https://tex.stackexchange.com/questions/577862
\newcommand\syllogism[3][]{%
    \begin{center}
        \def\tmp{#1}%
        \ifx\tmp\empty\else(#1)\quad\fi
        \begin{tabular}{@{}l@{}}
            #2\\\hline#3\quad$\therefore$
        \end{tabular}
    \end{center}
}
